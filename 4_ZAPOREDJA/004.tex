%priprava posamezne ure
%tukaj zaporedoma napisemo{st. zaporedne ure}{datum}{naslov}{poglavje}{oblika dela}{pripomocki}
\begin{priprava}{}{}{Zaporedja}{Geometrijsko zaporedje}{frontalna}{tabla}

\didopomba{Na enak način kot aritmetično zaporedje} Kaj je skupno naslednjim zaporedjem (konstanten količnik \didopomba{splošni člen poračunate potem pri vajah, razen če ga kdo sam ugane}):
\begin{align*}
    & 2, 6, 18, 54 \ldots && a_n = 2n+1 && d = 2 \\
    & -4, 8, -16, 32,  \ldots && a_n = -3n+11 && d = -3 \\
    & 8, 4, 2, 1, \frac{1}{2}, \frac{1}{4}, \frac{1}{8} \ldots && a_n = 5 && d = 0
\end{align*}

Zaporedje je \emph{geometrijsko}, kadar je količnik med zaporednimi členi konstanten.
$$ \frac{a_n}{a_{n-1}} = q \text{ za vsak } n \in \NN $$

Za splošno zaporedje $ a_1, a_2, a_3, a_4 \ldots, a_n \ldots $ lahko zapišemo splošni člen:
\begin{align*}
    & a_1 \\
    & a_2 = a_1 \cdot q \\
    & a_3 = a_2 \cdot q = a_1 \cdot q^2 \\
    & a_4 = a_3 \cdot q = a_1 \cdot q^3 \\
    & \vdots \\
    & \bf{a_n = a_1 \cdot q^{n-1}}
\end{align*}

Kaj nam $ q $ pove? \didopomba{Tu naj primere predlagajo sami}

\begin{center}
    \begin{tabular}{ |c|c|c| } 
     \hline
     & $ a_1 > 0 $ & $ a_1 < 0 $ \\
     \hline
     $ q > 1 $ & naraščajoče (npr. $ 2, 6, 18, 54 \ldots $) & padajoče (npr. $ -2, -6, -18, -54 \ldots $) \\
     \hline 
     $ 0 < q < 1 $ & padajoče (npr. $ 1, \frac{1}{2}, \frac{1}{4}, \frac{1}{8} \ldots $) & naraščajoče (npr. $ -1, -\frac{1}{2}, -\frac{1}{4}, -\frac{1}{8} \ldots $) \\ 
     \hline
     $ q < 0 $ & \multicolumn{2}{|c|}{alternijajoče/spreminjajoče (npr. $ -4, 8, -16, 32,  \ldots $)} \\
     \hline
    \end{tabular}
    \end{center}

\newpage

\vaje{
Vaja:
\begin{itemize}
    \item Izračunamo splošni člen s to formulo za zgornje tri primere. Ali so zaporedja naraščajoča, padajoča \ldots?
    \item Iz zaporedja splošni člen in $ q $ \didopomba{Life hack: $ a_1, a_2, a_3 $ zaporedni členi geometrijskega zaporedja: namesto $ a_1, a_2 = a_1q, a_3 = a_1q^2 $ gledaš $ a_1 = \frac{a_2}{q}, a_2, a_3 = a_2q $.}
    \item Iz splošnega člena prvih nekaj členov in $(2n+1)$-ti člen ali kaj podobnega
    \item Naj bodo $ a, x, b $ zaporedni členi geometrijskega zaporedja. Koliko je $ x $?
    \subitem \textcolor{black}{$ \frac{x}{a} = \frac{b}{x} \rightarrow x = \sqrt{ab} $ je \textbf{geometrična sredina} števil $ a $ in $ b $.}
    \item Dokaz, da je aritmetična sredina večja od geometrijske ter kdaj je enaka.
    \item Določi $ x $, da bodo to trije zaporedni členi geometrijskega zaporedja: $ 27^{3-x}, 3^{x-4}, 9^{2x-7} $ ALI $ \log_5(2x+5) + 2, 4\log_5 125, 7\log_2 64 - 6 $
    \item Ali obstaja zaporedje, ki je hkrati geometrijsko in aritmetično? \didopomba{da, konstantno, in to je edino}
\end{itemize}
}

\naslov{Vsota aritmetičnega zaporedja}

\begin{equation*}
    \begin{split}
    S_n & = \sum_{i=1}^{n} a_i = a_1 + a_2 + \ldots + a_n = \\
    & = a_1 + a_1 \cdot q + \ldots + a_1 \cdot q^{n-1} = \\
    & = a_1 (1 + q + q^2 + \ldots + q^n) = \\
    & = a_1 \frac{q^n - 1}{q - 1}, q \neq 1!
    \end{split}
\end{equation*}
\didopomba{Razmislek za $ q = 1 $ -- konstantno zaporedje, vsota je kar $ n \cdot a_1 $}

\vaje{
Vaje:
\begin{itemize}
    \item Vsota nekega zaporedja
    \item Izračunaj primeren $ x $: $ 5 + 15 + 45 + \ldots + x = 49205 $.
    \item Rešitvi spodnjih enačb sta zaporedoma prvi člen in količnik geometrijskega zaporedja. Najmanj koliko členov tega zaporedja moramo sešteti, da dobimo vsaj 7,83? \\
    $ \log_2 x = 2 - \log_2 (x-3) $ \\
    $ 5^{x+1} = 5 \sqrt{5} $
\end{itemize}
}
\end{priprava}