%priprava posamezne ure
%tukaj zaporedoma napisemo{st. zaporedne ure}{datum}{naslov}{poglavje}{oblika dela}{pripomocki}
\begin{priprava}{}{}{Zaporedja}{Obrestni račun}{frontalna}{tabla}

% Jaz bi to ločila na svoje poglavje - Finančna matematika:)

Razlaga izrazov glavnica, obrestna mera ... Pač osnovno o bančništvu \dots


NAVADNO OBRESTOVANJE

$ G_n = G_0 + G_0 \frac{p}{100} n $

OBRESTNO OBRESTOVANJE

$ G_n = G_0 r^n, r = 1 + \frac{p}{100} $

Primerjava z novčičem, kako je eksponentna bolj naraščujoča ...

OBROČNO ODPLAČEVANJE (VPLAČILA IN IZPLAČILA)

n let vplačujemo obroke $ a $. Koliko se nam nabere na koncu na banki? (s premico)

$ S_n = ar^{n-1} + \ldots + a = \frac{a(r^n - 1)}{r - 1} $

Izposodimo si $ G $ denarja, ki ga odplačujemo $ n $ let. Koliko mora znašati vsak obrok $ d $?

$ G r^n =  dr^{n-1} + \ldots + d \rightarrow d = G\frac{r^n(r - 1)}{r^n - 1} $

NAČELO EKVIVALENCE GLAVNICE - relativna obrestna mera

$ r_{(n)} = 1 + \frac{p/n}{100} $

Vaje:
\begin{itemize}
    \item
\end{itemize}

\end{priprava}