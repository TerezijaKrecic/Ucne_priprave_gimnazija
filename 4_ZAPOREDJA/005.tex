%priprava posamezne ure
%tukaj zaporedoma napisemo{st. zaporedne ure}{datum}{naslov}{poglavje}{oblika dela}{pripomocki}
\begin{priprava}{}{}{Zaporedja}{Popolna ali matematična indukcija}{frontalna}{tabla}

Koliko je vsota prvih $ n $ lihih števil ($ S_n $)?
\begin{equation*}
    \begin{split}
S_1 & = 1 \\
S_2 & = 1 + 3 = 4 \\
S_3 & = 1 + 3 + 5 = 9 \\
S_4 & = 1 + 3 + 5 + 7 = 16 \\
S_5 & = 25 \\
\vdots \\
S_n & = 1 + 3 + 5 + 7 + \ldots + 2n - 1
    \end{split}
\end{equation*}
Zdi se nam, da je $ S_n = n^2 $. Kako to preverimo?

\textbf{Popolna ali matematična indukcija} je način dokazovanja iz posameznega v splošno (najpogosteje pri naravnih številih). Poteka v dveh korakih:
\begin{enumerate}
    \item korak: preverimo, ali trditev velja za $ n = 1 $
    \item korak (\textbf{indukcijski korak}): ob predpostavki, da velja za $ n $, je treba dokazati, da velja za $ n + 1 $
\end{enumerate}

Preverimo pri zgornjem primeru:
\begin{enumerate}
    \item $ n = 1 $: $ S_1 = 1 = 1^2 $, formula drži
    \item $ n \rightarrow n + 1 $:
        \subitem indukcijska predpostavka: $ S_n = n^2 $ (predpostavimo, da to velja)
        \subitem indukcijski korak: $ S_{n+1} = (n+1)^2 $ (to želimo pokazati!)
        \subitem $ S_{n+1} = S_n + a_{n+1} = n^2 + (2n + 1) = (n + 1)^2 $, s tem smo dokazali pravilnost formule
\end{enumerate}

\vaje{
Vaje:
\begin{itemize}
    \item Trdim, da je $ \frac{1}{1 \cdot 2} + \frac{1}{2 \cdot 3} + \ldots + \frac{1}{n(n+1)} = \frac{n}{n+1} $. Dokaži, da je to res.
    \item Dokaži, da 31 deli števila oblike $ 5^{n+2} + 6^{2n+1} $
\end{itemize}
}

\didopomba{PAZLJIVOST -- lahko za foro si pogledamo primer -- kao paziti, kaj vzeti za prvi korak, da je smiselno} (``vsaka neprazna množica ima enake elemente'' -- če za prvi korak vzameš množico z enim elementom, trditev potem za splošno množico drži. Problem je dejansko pri množici z dvema elementoma)

\end{priprava}