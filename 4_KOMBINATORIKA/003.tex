%priprava posamezne ure
%tukaj zaporedoma napisemo{st. zaporedne ure}{datum}{naslov}{poglavje}{oblika dela}{pripomocki}
\begin{priprava}{}{}{Kombinatorika}{Variacije}{frontalna}{tabla}

Uvodna motivacija: V vrečki imamo 5 žog različnih barv. Trikrat zapored izvečemo žogo iz vrečke in jo postavimo v vrsto. Koliko barvnih vzorcev lahko dobimo? \didopomba{Za prvo mesto je 5 možnosti, za drugo 4 ter za tretje 3, torej $ 5 \cdot 4 \cdot 3  = $ (NAMIG) $ = \frac{5!}{2!} = \frac{5!}{(5-2)!} $}

\textcolor{red}{\textbf{Variacija reda $ r $ med $ n $ elementi} = razporeditev $ n $ različnih elementov na $ r $ mest, kjer je $ r < n $.}
Število teh razporeditev običajno označimo z $ V^r_n $.
$$ V^r_n = \frac{n!}{(n - r)!} = n \cdot (n - 1) \cdot (n - 2) \cdot \ldots \cdot (n - r + 1) $$

\textcolor{red}{\textbf{Variacija s ponavljanjem}} = razporeditev, kjer se elementi lahko ponavljajo.

Vzamemo vrečko z žogami iz prve naloge in trikrat izvlečemo žogo, ampak jo takoj nato vrnemo. Koliko vzorcev lahko dobimo? \didopomba{Za prvo mesto je 5 možnosti, za drugo tudi 5 (ker smo prvo žogo vrnili) in za tretje spet 5, skupaj $ 5 \cdot 5 \cdot 5 = 5^3 \rightarrow $ naj sklepajo na formulo}
$$ ^{(p)}V^r_n = n^r $$


\vaje{
Vaje:
\begin{itemize}
    \item vsega sorte
\end{itemize}
}
    
\end{priprava}