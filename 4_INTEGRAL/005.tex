%priprava posamezne ure
%tukaj zaporedoma napisemo{st. zaporedne ure}{datum}{naslov}{poglavje}{oblika dela}{pripomocki}
\begin{priprava}{}{}{Integral}{Uporaba v fiziki}{frontalna}{tabla}

\didopomba{Začnemo s konstantno hitrostjo, narišemo graf, koliko je pot? Ploščina pravokotnika. Kaj pa, če je hitrost odsekoma konstantna? Še vedno ploščina, pač sešteješ ploščine stolpcev. Kaj pa, če je vijuga? Dobimo isto slikco kot pri določenem integralu, waw, pot je v bistvu določen integral hitrosti}


Kolikšno pot prepotujemo v $ 10 s $, če se v času $ t = 0 $ začnemo gibati s hitrostjo, ki jo opisuje funkcija $ v(t) = t^2 $?

\didopomba{R: $ s = \int_0^{10} t^2 dt $.}

Ali pa, da je $ a = \frac{dv}{dt}, v = \frac{ds}{dt} $.
    
\end{priprava}