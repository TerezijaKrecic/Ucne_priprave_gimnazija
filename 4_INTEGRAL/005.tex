%priprava posamezne ure
%tukaj zaporedoma napisemo{st. zaporedne ure}{datum}{naslov}{poglavje}{oblika dela}{pripomocki}
\begin{priprava}{}{}{Integral}{Uporaba v fiziki}{frontalna}{tabla}

\didopomba{Izbirno}

Naj pogruntajo (saj mogoče že vedo), da formula $ s = v \dot t $ izhaja iz ploščine pravokotnika pod grafom, kjer je v konstantna funkcija časa. Torej če v ni konstantna, je pot kar integral!

Kolikšno pot prepotujemo v $ 10 s $, če se v času $ t = 0 $ začnemo gibati s hitrostjo, ki jo opisuje funkcija $ v(t) = t^2 $?

\didopomba{R: V nekem kratkem časovnem intervalu $ dt $ je trenutna hitrost praktično konstantna, zato v tem času prepotujemo $ v(t) dt $ poti. Celotna pot je potem integral teh majhnih poti, $ s = \int_0^{10} t^2 dt $.}

Ali pa, da je $ a = \frac{dv}{dt}, v = \frac{ds}{dt} $.
    
\end{priprava}