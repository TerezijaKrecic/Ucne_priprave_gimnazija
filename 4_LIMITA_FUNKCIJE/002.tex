%priprava posamezne ure
%tukaj zaporedoma napisemo{st. zaporedne ure}{datum}{naslov}{poglavje}{oblika dela}{pripomocki}
\begin{priprava}{}{}{Limita funkcije}{Zveznost funkcije}{frontalna}{tabla}

Funkcija je \textbf{zvezna}, če je njen graf neprekinjena črta (linearna funkcija, kvadratna, polinomi, logaritemska \ldots). \didopomba{torej pri zveznosti ne gledaš $ D_f $, ampak celo realno os!}

Racionalna funkcija ni zvezna v polih.

\vaje{
Vaje:
\begin{itemize}
    \item Risanje kosoma definiranih funkcij in ugotavljanje točk nezveznosti. \didopomba{tudi, če se črta grafa ``le'' zlomi, je funkcija tam zvezna}
\end{itemize}
}

\didopomba{najprej tako definiramo zveznost, ker jim je najbolj enostavna za razumeti, brez tistih limit. Sedaj gremo na limite in potem tam omenimo še enkrat zveznost.}
    
\end{priprava}