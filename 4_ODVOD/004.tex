%priprava posamezne ure
%tukaj zaporedoma napisemo{st. zaporedne ure}{datum}{naslov}{poglavje}{oblika dela}{pripomocki}
\begin{priprava}{}{}{Odvod}{Uporaba drugih odvodov}{frontalna}{tabla}

\didopomba{Najprej -- grafična predstava odvoda: nekaj primerov grafov funkcij, pod njimi pa grafi odvodov. Npr. za $ x^2 $ je fora, da je na levi strani graf negativen, v 0 je 0, na desni pozitiven. Ni nujno, da je premica (seveda, ko izračunamo, pač je, ampak za slikco lahko določimo le predznak!)}

Drugi odvod = odvod funkcije še enkrat odvajaš: $ f''(x) $.

Naredimo tri slikice druga pod drugo. na vrhnji je prikazana $ f $, pod njo $ f' $ in na spodnji $ f'' $. Primerjamo in ugotavljamo:

\begin{itemize}
    \item Funkcija je \textbf{konkavna} za vse $ x $, za katere velja $ f''(x) < 0 $.
    \item Funkcija je \textbf{konveksna} za vse $ x $, za katere velja $ f''(x) > 0 $.
\end{itemize}

Stacionarne točke (torej $ f'(x) = 0 $):
\begin{itemize}
    \item če $ f''(x) < 0 \rightarrow $ maksimum
    \item če $ f''(x) > 0 \rightarrow $ minimum
    \item če $ f''(x) = 0 \rightarrow $ prevoj
\end{itemize}

\vaje{
Vaje:
\begin{itemize}
    \item 
\end{itemize}
}
    
\end{priprava}